% documentclass: article used for scientific journals, short reports, program documentation, etc
% options: fontsize 11, generate document for double sided printing, a4-paper
\documentclass[10pt, twoside, a4paper]{article}

% package for changing page layout
\usepackage{geometry}
\geometry{a4paper, lmargin=40mm, rmargin=45mm, tmargin=40mm, bmargin=45mm}
% set indentation
\setlength{\parindent}{5mm} 

% package for changing page layout (used to indent whole paragraphs)
\usepackage{changepage}

% input encoding for special characters (e.g. ä,ü,ö,ß), only for non english text
% options: utf8 as encoding standard
\usepackage[utf8]{inputenc}
% package for changing used language
% options: german or default: english
\usepackage[german]{babel}

% package for font encoding
\usepackage[T1]{fontenc}
% package for times font
% \usepackage{times}
% package for latin modern fonts
\usepackage{lmodern}

% package for math symbols, functions and environments from ams(american mathematical society)
\usepackage{amsmath, mathtools}
% \usepackage{mathtools}
% package for extended symbols from ams
\usepackage{amssymb}
\usepackage{mathrsfs}
% package for extended symbols from stmaryrd(st mary road)
\usepackage{stmaryrd}
% package for managing pictures
\usepackage{graphicx}
%package for positioning figures
\usepackage{float}
% package for changing color of font and paper
% options: using names of default colors (e.g red, black)
% \usepackage[usenames]{color}
\usepackage[dvipsnames]{xcolor}

% package for customising captions
\usepackage[footnotesize, hang]{caption}

% package for creating framed, shaded, or differently highlighted regions that can break across pages; environments: framed, oframed, shaded, shaded*, snugshade, snugshade*, leftbar, titled-frame
\usepackage{framed}

% unnumerated definition environment definiton
\newenvironment{definition*}[2]{
    \begin{framed}
    \noindent
    { \fontfamily{ptm}\selectfont \textsc{\textbf{#1:}} } ~ #2 
    \par \hfill\\ 
    \fontfamily{lmr}\selectfont \itshape
}{
    \end{framed}
}
% definitions for numerated definition environment
\newcounter{definition}[section]
\newcommand*{\definitionnum}{\thesection.\arabic{definition}}
\newenvironment{definition}[2]{
    \refstepcounter{definition}
    \begin{definition*}{#1 \definitionnum}{#2}
}{
    \end{definition*}
}

% unnumerated theorem environment definition
\newenvironment{theorem*}[2]{
    \begin{leftbar}
    \noindent
    { \fontfamily{ptm}\selectfont \textbf{\textsc{#1}:}} ~ #2 
    \par \hfill\\ 
    \fontfamily{lmr} \fontshape{it} \selectfont
}{ 
    \end{leftbar}
}
% definitions for numerated theorem environment
\newcounter{theorem}[section]
\newcommand*\theoremnum{\thesection.\arabic{theorem}}
\renewcommand\thetheorem{\theoremnum}
\newenvironment{theorem}[2]{
    \refstepcounter{theorem}
    \begin{theorem*}{#1 \theoremnum}{#2}
}{
    \end{theorem*}
}

% change enumeration style of equations
% \renewcommand\theequation{\thesection.\arabic{equation}}

% q.e.d. definition
\newcommand{\qed}{ \par \hfill \fontfamily{lmr} \fontshape{it} \selectfont \mbox{q.e.d.} \\}
\newcommand{\qedbox}{ \par \hfill $\Box$ \\ }

% proof environment definition for theorems
\newenvironment{proof}{
    \par \hfill\\
    \noindent
    { \fontfamily{ptm}\selectfont \textsc{Beweis:}} 
    \normalfont
    \small
    \begin{adjustwidth}{8mm}{} 
}{ 
    \end{adjustwidth} 
    \qed
}


% new displaymath command, so that equations will not be stretched
\newcommand{\D}[1]{\mbox{$ #1 $}}
% make unnumerated equation
\newcommand{\E}[1]{\[ #1 \]}
% command for set/family
\newcommand{\set}[1]{\left{ #1 \right}}
% command for box brackets
\newcommand{\boxb}[1]{\left[ #1 \right]}
% command for parentheses/curved brackets
\newcommand{\curvb}[1]{\left( #1 \right)}
% command for angle brackets
\newcommand{\angleb}[1]{\left\langle #1 \right\rangle}
%
\newcommand{\diff}{\mathrm{d}}
% mathematical definitions (standard sets)
\newcommand{\SR}{\mathbb{R}}
\newcommand{\SRP}{\SR^+}
\newcommand{\SRPN}{\SRP_0}
\newcommand{\SN}{\mathbb{N}}
\newcommand{\SNN}{\mathbb{N}_0}
\newcommand{\SZ}{\mathbb{Z}}
\newcommand{\SQ}{\mathbb{Q}}
\newcommand{\SQP}{\mathbb{Q}^+}
\newcommand{\SQPN}{\mathbb{Q}^+_0}


% package for init listings(non-formatted  text) e.g. different source codes
\usepackage{listings}
% declaring new caption format for listings
\DeclareCaptionFormat{listing}{
    \par\framebox[\textwidth]{ {\bfseries #1#2} \hfill \itshape #3 }
    \vspace{-0.9\baselineskip}
}
\captionsetup[lstlisting]{format=listing, singlelinecheck=off}
\definecolor{codeDarkGray}{gray}{0.2}

\newcounter{lstlistingnum}[section]
% style definition for standard code listings
\lstdefinestyle{std}{
    belowcaptionskip=0.5\baselineskip,
    breaklines=true,
    frame=tB,
    xleftmargin=0mm,
    showstringspaces=false,
    basicstyle= \footnotesize\ttfamily,
    keywordstyle= \color{codeDarkGray}\bfseries,
    commentstyle= \itshape\color{gray},
    identifierstyle=\color{codeDarkGray},
    stringstyle=\color{codeDarkGray},
    numberstyle=\scriptsize\ttfamily,
    numbers=left,
    captionpos=t,
    tabsize=4
}
% command for standard code listings (unnumerated)
\newcommand{\inputcodes}[3][Listing]{
    \renewcommand{\thelstlisting}{}
    \renewcommand{\lstlistingname}{#1}
    \lstinputlisting[style=std, #2]{#3}
}
% command for standard code listings (numerated)
\newcommand{\inputcode}[3][Listing]{
    \refstepcounter{lstlistingnum}
    \renewcommand{\thelstlisting}{\thesection.\arabic{lstlistingnum}}
    \renewcommand{\lstlistingname}{#1}
    \lstinputlisting[style=std, #2]{#3}
}
% command for c++ code listings in color
\newcommand{\inputcppc}[3][Quelltext]{
    \inputcode[#1]{language=C++, morekeywords = {__global__, __kernel__}, keywordstyle = \bfseries\color{MidnightBlue}, stringstyle=\color{BurntOrange}, caption = #2}{#3}
}
\newcommand{\inputcppcs}[3][Quelltext]{
    \inputcodes[#1]{language=C++, morekeywords = {__global__, __kernel__}, keywordstyle = \bfseries\color{MidnightBlue}, stringstyle=\color{BurntOrange}, caption = #2}{#3}
}
% command for c++ code listings in grayscale
\newcommand{\inputcpp}[3][Quelltext]{
    \inputcode[#1]{language=C++, morekeywords={__global__, __kernel__}, caption = #2}{#3}
}
\newcommand{\inputcpps}[3][Quelltext]{
    \inputcodes[#1]{language=C++, morekeywords={__global__, __kernel__}, caption = #2}{#3}
}


% package for including csv-tables from file
% \usepackage{csvsimple}
% package for creating, loading and manipulating databases
\usepackage{datatool}

% package for converting eps-files to pdf-files and then include them
\usepackage{epstopdf}
% use another program (ps2pdf) for converting
% !!! important: set shell_escape=1 in /etc/texmf/texmf.cnf (Linux/Ubuntu 12.04) for allowing to use other programs
% !!!            or use the command line with -shell-escape
\epstopdfDeclareGraphicsRule{.eps}{pdf}{.pdf}{
ps2pdf -dEPSCrop #1 \OutputFile
}


% package for reference to last page (output number of last page)
\usepackage{lastpage}
% package for using header and footer
% options: automate terms of right and left marks
\usepackage[automark]{scrpage2}
% \setlength{\headheight}{4\baselineskip}
% set style for footer and header
\pagestyle{scrheadings}
% \pagestyle{headings}
% clear pagestyle for redefining
\clearscrheadfoot
% set header and footer: use <xx>head/foot[]{Text} (i...inner, o...outer, c...center, o...odd, e...even, l...left, r...right)
\ihead[]{Thermodynamik - Übung 08 \\ Übung: Mo 10-12}
\ohead[]{Markus Pawellek \\ 144645}
\cfoot[]{\newline\newline\newline\pagemark}
% use that for mark to last page: \pageref{LastPage}
% set header separation line
\setheadsepline[\textwidth]{0.5pt}
% set foot separation line
\setfootsepline[\textwidth]{0.5pt}


\title{Thermodynamik - Übung 08}
\author{Markus Pawellek}
% \date{}


\section*{\centering Thermodynamik - Übung 10} % (fold)
\label{sec:thermodynamik_bung_10}

	\subsection*{Aufgabe 1} % (fold)
\label{sub:aufgabe_1}

	Sei $n\in\SN$ mit $n\geq 2$ die betrachtete Dimension des Raumes.
	Sei dann die allgemeine Koordinatentransformation von Kugelkoordinaten auf kartesische Koordinaten definiert durch
	\footnote{
		Die Idee hierfür kommt durch eine induktive Verallgemeinerung von zwei- und dreidimensionalen Kugelkoordinaten, die hier nicht erklärt werden soll.
		(Sie steht in etwas längerer Form an meinem Whiteboard.)
	}
	\[ \Phi^{n}:[0,\infty)\times[-\pi,\pi]\times\left[-\frac{\pi}{2},\frac{\pi}{2}\right]^{n-2}\longrightarrow \SR^n \]
	\begin{alignat}{3}
		&x_i:=\Phi^{n}_i (r,\alpha_1,\ldots,\alpha_{n-1}) &&:= r\sin \alpha_{i-1} &&\prod_{k=i}^{n-1} \cos \alpha_k \quad \textit{für alle } k\in\curlb{2,\ldots,n} \nonumber \\
		&x_1:=\Phi^{n}_1 (r,\alpha_1,\ldots,\alpha_{n-1}) &&:= r\cos \alpha_{1} &&\prod_{k=2}^{n-1} \cos \alpha_k \nonumber
	\end{alignat}
	Die Koordinatentransformation führt den Winkel $\vartheta$ der dreidimensionalen Kugelkoordinaten anders ein, als bekannt.
	Er wird hier um $\pi/2$ verschoben, um so einen einfacheren induktiven Vorgang zu ermöglichen.
	Die angegebenen Funktionen lassen sich außer auf einer Lebesgue-Nullmenge umkehren.
	Dabei entstehen durch Umformung die Gleichungen
	\begin{alignat}{2}
		r &= \sqrt{\sum_{i=1}^n x_i^2} \nonumber \\
		\alpha_i &= \arcsin\curvb{ \frac{x_{i+1}}{\sqrt{\sum_{k=1}^{i+1} x_k^2}} } \quad \textit{für alle } k\in\curlb{2,\ldots,n} \nonumber \\
		\alpha_1 &= 
			\begin{cases}
				\arcsin\curvb{ \frac{x_2}{\sqrt{ x_1^2 + x_2^2 }} } & :x_1 > 0 \\ 
				\arccos\curvb{ \frac{x_2}{\sqrt{ x_1^2 + x_2^2 }} } + \dfrac{\pi}{2} & :x_1 < 0,\, x_2 > 0 \\
				\arccos\curvb{ \frac{-x_2}{\sqrt{ x_1^2 + x_2^2 }} } -\dfrac{\pi}{2} & :x_1 < 0,\, x_2 < 0 
			\end{cases}
		\nonumber
	\end{alignat}
	Dies lässt sich durch vollständige Induktion nachweisen.
	Um nun die Integration über diese Koordinaten zu ermöglichen, muss die Determinante der Jacobi-Matrix von $\Phi^n$ berechnet werden.
	Dazu muss als erstes die Jacobi-Matrix dargestellt werden.
	Verwendet man ein ähnliches induktives Vorgehen, kommt man für alle $n\in\SN, n\geq2$ zu
	\[
		\Diff\Phi^{n+1}(\ldots) = 
		\left(
		\begin{array}{c|c}
			\cos\alpha_n\cdot\Diff\Phi^n(\ldots) 
			& 
			\begin{array}{c}
				-r\sin\alpha_n\cdot\Diff\Phi_{11}^n(\ldots) \\
				\vdots \\
				-r\sin\alpha_n\cdot\Diff\Phi_{n1}^n(\ldots) \\
			\end{array}\\
			\hline
			\begin{array}{rrrr}
				\sin\alpha_n & 0 & \cdots & 0 \\
			\end{array} 
			& 
			r\cos\alpha_n
		\end{array}
		\right)
	\]
	Entwickelt man nun die Determinante nach der letzten Zeile (dies ist naheliegend, da hier nur zwei Terme vorkommen, welche ungleich sind) folgt
	\begin{alignat}{2}
		\det \Diff\Phi^{n+1}(\ldots) = &(-1)^n\, r \cos\alpha_n \cdot \cos^{n}\alpha_n \, \det\Diff\Phi^n(\ldots) \nonumber \\
		& + \sin\alpha_n \cdot (-1)^{n-1} \, (-r) \, \sin\alpha_n \, \cos^{n-1}\alpha_n \, \det\Diff\Phi^n(\ldots) \nonumber
	\end{alignat}
	Für den zweiten Summanden wurde die rechte Spalte der Untermatrix zirkulär nach links vertauscht.
	So entsteht ein $(-1)^{n-1}$.
	Sowohl für die eine als auch die andere Untermatrix wurde der Satz zur Berechnung einer Determinanten, deren Spalten mit verschiedenen Skalaren multipliziert wurden, verwendet.
	So entstehen die Vorfaktoren $\cos^{n}\alpha_n$ und $(-r)\,\sin\alpha_n \, \cos^{n-1}\alpha_n$.
	Weiteres Umformen ergibt dann die Rekursionsformel der Determinanten und ein Rekursionsbeginn für $n = 2$, sodass für alle $n\in\SN, n\geq2$ gilt
	\begin{alignat}{2}
		&\det \Diff\Phi^{n+1}(\ldots) &&= (-1)^n \ r \ \cos^{n-1}\alpha_n \ \det\Diff\Phi^n(\ldots) \nonumber \\
		&\det \Diff\Phi^{2}(\ldots) &&= r \nonumber
	\end{alignat}
	Die Lösung der Rekursionsformel erfolgt durch Entwickeln und vollständige Induktion (dies wird hier nicht ausgeführt).
	Es folgt dann direkt
	\[
		\abs{ \det\Diff\Phi^{n}(\ldots) } = r^{n-1} \ \prod_{i=1}^{n-2} \cos^i\alpha_{i+1}
	\]
	Seien nun $B_R := \set{ x \in \SR^n }{ \norm{x} \leq R }$ für ein $R\in[0,\infty)$, $\lambda$ das Lebesgue-Maß und $\sigma$ das Oberflächenmaß.
	Dann gilt nach dem Satz über Koordinatentransformationen und dem Satz von Fubini/Tonelli
	\begin{alignat}{2}
		V := \lambda(B_R) = \integral{B_R}{}{}{\lambda} &= \int_0^R \int_{-\pi}^\pi \int_{-\pi/2}^{\pi/2}\cdots\int_{-\pi/2}^{\pi/2} \abs{ \det\Diff\Phi^{n}(\ldots) } \, \diff\alpha_{n-1}\cdots\diff\alpha_1\diff r \nonumber \\
		&= \dfrac{2\pi}{n}R^n \prod_{j=1}^{n-2} \integral{-\pi/2}{\pi/2}{\cos^j\alpha}{\alpha} \nonumber \\
		A := \sigma(B_R) = \integral{B_R}{}{}{\sigma} &= \int_{-\pi}^\pi \int_{-\pi/2}^{\pi/2}\cdots\int_{-\pi/2}^{\pi/2} \abs{ \det\Diff\Phi^{n}(R,\ldots) } \, \diff\alpha_{n-1}\cdots\diff\alpha_1 \nonumber \\
		&= 2\pi R^{n-1} \prod_{j=1}^{n-2} \integral{-\pi/2}{\pi/2}{\cos^j\alpha}{\alpha} \nonumber
	\end{alignat}
	Weiterhin gilt im Allgemeinen folgende Rekursionsformel für alle $n\in\SN, n>2$
	\[ \integral{-\pi/2}{\pi/2}{\cos^n\alpha}{\alpha} = \dfrac{n-1}{n} \integral{-\pi/2}{\pi/2}{\cos^{n-2}\alpha}{\alpha} \]
	Es folgt dann für alle $k\in\SN$
	\begin{alignat}{3}
		&\integral{-\pi/2}{\pi/2}{\cos^{2k}\alpha}{\alpha} &&= \dfrac{\pi}{2} &&\prod_{m=1}^{k-1} \dfrac{2m+1}{2(m+1)} \nonumber \\
		&\integral{-\pi/2}{\pi/2}{\cos^{2k-1}\alpha}{\alpha} &&= 2 &&\prod_{m=1}^{k-1} \dfrac{2m}{2m+1} \nonumber \\
		&\integral{-\pi/2}{\pi/2}{\cos\alpha}{\alpha} &&= 2 \nonumber \\
		&\integral{-\pi/2}{\pi/2}{\cos^{2}\alpha}{\alpha} &&= \dfrac{\pi}{2} \nonumber
	\end{alignat}

	\paragraph{Fall $n=2k-1$ für ein $k\in\SN$:}
	\begin{alignat}{2}
		&\prod_{j=1}^{n-2} \integral{-\pi/2}{\pi/2}{\cos^j\alpha}{\alpha} \nonumber \\
		&= \integral{-\pi/2}{\pi/2}{\cos^{2k-3}\alpha}{\alpha} \boxb{ \prod_{m=1}^{k-2} \integral{-\pi/2}{\pi/2}{\cos^{2m-1}\alpha}{\alpha} \integral{-\pi/2}{\pi/2}{\cos^{2m}\alpha}{\alpha} } \nonumber \\ 
		&= 2 \boxb{ \prod_{m=1}^{k-2} \dfrac{2m}{2m+1} } \boxb{ \prod_{m=1}^{k-2} \pi \prod_{j=1}^{m-1} \dfrac{2j}{2j+1}\dfrac{2j+1}{2(j+1)} } \nonumber \\
		&= 2^{k-1}\pi^{k-2} \boxb{ \prod_{m=1}^{k-2} \dfrac{m}{2m+1} } \cdot \dfrac{1}{(k-2)!} \nonumber \\
		&= 2^{k-1}\pi^{k-2} \prod_{m=1}^{k-2} \dfrac{1}{2m+1} \nonumber
	\end{alignat}

	\paragraph{Fall $n=2k$ für ein $k\in\SN$:}
	\begin{alignat}{2}
		\prod_{j=1}^{n-2} \integral{-\pi/2}{\pi/2}{\cos^j\alpha}{\alpha} &= \prod_{m=1}^{k-1} \integral{-\pi/2}{\pi/2}{\cos^{2m-1}\alpha}{\alpha} \integral{-\pi/2}{\pi/2}{\cos^{2m}\alpha}{\alpha} \nonumber \\
		&= \dfrac{\pi^{k-1}}{(k-1)!} \nonumber
	\end{alignat}

	Damit wurden nun die unbekannten Faktoren in den gesuchten Gleichungen bestimmt.
	Das allgemeine Volumen und der allgemeine Oberflächeninhalt folgt nun durch Einsetzen.
	\begin{alignat}{2}
		V &=
		\begin{cases}
			\frac{\pi^k}{k!}R^n & :n=2k, k\in\SN \\
			2^{k}\pi^{k-1}R^n \prod_{m=1}^{k-1} \frac{1}{2m+1} & :n=2k-1, k\in\SN \\
		\end{cases}
		\nonumber \\
		A &=
		\begin{cases}
			\frac{2\pi^k}{(k-1)!}R^{n-1} & :n=2k, k\in\SN \\
			2^{k}\pi^{k-1}R^{n-1} \prod_{m=1}^{k-2} \frac{1}{2m+1} & :n=2k-1, k\in\SN \\
		\end{cases}
		\nonumber
	\end{alignat}
	\newpage
	Sei nun $R>\varepsilon > 0$.
	Dann besitzt die äußere Kugelschale mit der Schichtdicke $\varepsilon$ das Volumen $\Delta V$, wenn $\xi$ die Funktion der Vorfaktoren des Volumens darstellt.
	\begin{alignat}{2}
		&\Delta V &&= \xi(n) \boxb{ R^n - (R-\varepsilon)^n } \nonumber \\
		\Rightarrow &\frac{\Delta V}{V} &&= 1 - \underbrace{\curvb{ 1-\frac{\varepsilon}{R} }^n }_{\mathclap{\longrightarrow 0, n\longrightarrow \infty}} \longrightarrow 1, \ n\longrightarrow\infty \nonumber
	\end{alignat}
	Damit befindet sich für große Dimensionen $n$ und für alle $\varepsilon>0$ fast das gesamte Volumen in der äußeren Schale.

% subsection aufgabe_1 (end)
	\newpage
	\subsection*{Aufgabe 2} % (fold)
\label{sub:aufgabe_2}

	Sei $n\in\SN$ mit $n>1$.
	Sei weiterhin \D{Z=(1,2,\ldots,n)} eine Zerlegung von $[1,n]$.
	Die Funktion $\ln n!$ ist monoton steigend und positiv für alle $n\in\SN$ mit $n\geq 1$.
	Es gilt nach Gesetzen des Logarithmus
	\[ \ln n! = \sum_{i=1}^{n}\ln{i} \]
	Daraus folgt dann durch Betrachtung der Ober- und Untersummen
	\[ \integral{1}{n}{\ln x}{x} \ \leq \ \sum_{i=1}^n \ln i \ \leq \ \integral{1}{n+1}{\ln x}{x} \]
	Rückumformung und Anwendung des Hauptsatzes der Integral- und Differentialrechnung resultiert dann in
	\begin{alignat}{6}
		&&n\ln n - n+1 \ \leq && \centermathcell{\ln n!} &\leq \ (n+1)\ln(n+1) -n \nonumber \\
		\Rightarrow &&\ln n -\frac{n-1}{n} \ \leq &&\centermathcell{\frac{\ln n!}{n}} &\leq \ \frac{n+1}{n}\ln(n+1) - 1 \nonumber \\
		\Rightarrow &&\underbrace{-\frac{n-1}{n}}_{\mathclap{\longrightarrow -1,\ n \longrightarrow \infty}} \ \leq &&\centermathcell{\ \frac{\ln n!}{n} - \ln n} \ &\leq \ \underbrace{\frac{n+1}{n}\ln(n+1) -\ln n - 1}_{\mathclap{= \ln\frac{n+1}{n} + \frac{\ln(n+1)}{n} - 1 \longrightarrow -1,\ n \longrightarrow \infty}} \nonumber
	\end{alignat}
	Nach dem Sandwich-Theorem für Folgen folgt also
	\[ \frac{\ln n!}{n} - \ln n  \longrightarrow -1,\ n \longrightarrow \infty \]
	\qed

% subsection aufgabe_2 (end)
	\newpage
	\subsection*{Aufgabe 3} % (fold)
\label{sub:aufgabe_3}

	Seien $n\in\SN$ die Anzahl der Tage, an denen Personen Geburtstag haben können, und $k\in\SN$ mit $k<n$ die Anzahl der betrachteten Personen.
	Nimmt man die Gleichverteilung für Geburtstage an, so folgt für das Wahrscheinlichkeitsmaß
	\footnote{Der Wahrscheinlichkeitsraum soll hier nicht genauer angegeben werden.}
	\[ P:\SP\curvb{ \curlb{1,\ldots,n}^k }\longrightarrow [0,1] ,\qquad P((n_1,\ldots,n_k)) = \frac{1}{n^k} \]
	Das komplementäre Ereignis von 
	\[ \textit{Mindestens zwei Personen haben am gleichen Tag Geburtstag.} \]
	ist
	\[ \textit{Alle Personen haben an verschiedenen Tagen Geburtstag.} \]
	Dieses komplementäre Ereignis kann gerade durch folgende Menge beschrieben werden.
	$$ A:=\set{ (n_1,\ldots,n_k)\in\curlb{1,\ldots,n}^k }{ n_i \neq n_j \textit{ für alle } i,j\in\curlb{1,\ldots,k},i\neq j } $$
	Um nun die Anzahl der Elemente von $A$ zu bestimmen, bestimmt man induktiv die Anzahl der Möglichkeiten der einzelnen $n_i$.
	Beginnt man der Benennung entsprechend mit $n_1$, so können dessen Werte beliebig gewählt werden, da die anderen Werte noch nicht festgelegt wurden.
	Damit gibt es für $n_1$ also $n$ Möglichkeiten.
	Folglich kann aber $n_2$ nicht mehr beliebig gewählt werden, weil bereits $n_1$ einen der beliebigen $n$ Werte angenommen hat.
	Für $n_2$ gibt es also genau eine Lösung weniger und damit genau $n-1$ Möglichkeiten.
	Wählt man nun ein ganzzahliges $i<k$, sodass die Anzahl der Möglichkeiten für $n_i$ bekannt ist und mit $N_i$ bezeichnet wird, dann müssen sich die Möglichkeiten für $n_{i+1}$ durch $N_i -1$ ergeben, da durch $n_i$ einer der noch möglichen $N_i$ Werte besetzt wird.
	Die Anzahl der Gesamtmöglichkeiten ergibt sich dann durch Multiplikation der induktiv bestimmten Werte.
	$$ \#A = n\cdot(n-1)\cdot(n-2)\cdot\cdots\cdot(n-k+2)\cdot(n-k+1) = \frac{n!}{(n-k)!} $$
	Es folgt dann
	$$ P(A) = \frac{\#A}{n^k} = \frac{n!}{n^k(n-k)!} $$
	Nach Gesetzen der Wahrscheinlichkeitsrechnung  ergibt sich dann die Wahrscheinlichkeit des eigentlichen Ereignisses durch
	$$ \boxed{P(A^c) = 1- \frac{n!}{n^k(n-k)!}} $$
	\begin{description}
		\item[$n=365,k=20$:]
			$$ P(A^c) \approx 0.411 $$
		\item[$n=365,k=60$:]
			$$ P(A^c) \approx 0.994 $$ 
	\end{description}

% subsection aufgabe_3 (end)

% section thermodynamik_bung_10 (end)