% documentclass: article used for scientific journals, short reports, program documentation, etc
% options: fontsize 11, generate document for double sided printing, a4-paper
\documentclass[10pt, twoside, a4paper]{article}

% package for changing page layout
\usepackage{geometry}
\geometry{a4paper, lmargin=40mm, rmargin=45mm, tmargin=40mm, bmargin=45mm}
% set indentation
\setlength{\parindent}{1em}
% set factor for line spacing
% \linespread{1.0}\selectfont
% set (dynamic) additional line spacing
% \setlength{\parskip}{1ex plus 0.5ex minus 0.3ex}

% rigorous formatting (not too much hyphens)
% \fussy
% \sloppy

% package for changing page layout (used to indent whole paragraphs with adjustwidth)
\usepackage{changepage}

% input encoding for special characters (e.g. ä,ü,ö,ß), only for non english text
% options: utf8 as encoding standard, latin1
\usepackage[utf8]{inputenc}
% package for font encoding
\usepackage[T1]{fontenc}
% package for changing used language (especially for more than one language)
% options: ngerman (new spelling) or default: english
\usepackage[ngerman]{babel}
% package for times font
% \usepackage{times}
% package for latin modern fonts
\usepackage{lmodern}

% package for math symbols, functions and environments from ams(american mathematical society)
\usepackage{amsmath}
\usepackage{mathtools}
% package for extended symbols from ams
\usepackage{amssymb}
% package for math black board symbols (e.g. R,Q,Z,...)
\usepackage{bbm}
\usepackage{mathrsfs}
% package for extended symbols from stmaryrd(st mary road)
\usepackage{stmaryrd}

% package for including extern graphics plus scaling and rotating
\usepackage{graphicx}
%package for positioning figures
\usepackage{float}
% package for changing color of font and paper
% options: using names of default colors (e.g red, black)
% \usepackage[usenames]{color}
\usepackage[dvipsnames]{xcolor}
\definecolor{shadecolor}{gray}{0.9}
% package for customising captions
\usepackage[footnotesize, hang]{caption}
% package for customising enumerations (e.g. axioms)
\usepackage{enumitem}
% calc package reimplements \setcounter, \addtocounter, \setlength and \addtolength: commands now accept an infix notation expression
\usepackage{calc}
% package for creating framed, shaded, or differently highlighted regions that can break across pages; environments: framed, oframed, shaded, shaded*, snugshade, snugshade*, leftbar, titled-frame
\usepackage{framed}
% package for creating custom "list of"
% options: titles: do not intefere with standard headings for "list of"
\usepackage[titles]{tocloft}


% change enumeration style of equations
% \renewcommand\theequation{\thesection.\arabic{equation}}

% init list of math for definitions and theorems
\newcommand{\listofmathcall}{Verzeichnis der Definitionen und Sätze}
\newlistof{math}{mathlist}{\listofmathcall}
% add parentheses around argument
\newcommand{\parent}[1]{ \ifx&#1&\else (#1) \fi }
% unnumerated mathematical definition environment definiton
\newenvironment{mathdef*}[2]{
	\begin{framed}
	\noindent
	{ \fontfamily{ppl}\selectfont \textbf{\textsc{#1:}} } ~ #2 
	\par \hfill\\ 
	\fontfamily{lmr}\selectfont \itshape
}{
	\end{framed}
}
% definitions for numerated mathematical definition environment
\newcounter{mathdefc}[section]
\newcommand*{\mathdefnum}{\thesection.\arabic{mathdefc}}
\renewcommand{\themathdefc}{\mathdefnum}
\newenvironment{mathdef}[2]{
	\refstepcounter{mathdefc}
	\addcontentsline{mathlist}{figure}{\protect{\numberline{\mathdefnum}#1 ~ #2}}
	\begin{mathdef*}{#1 \mathdefnum}{#2}
}{
	\end{mathdef*}
}
% standard mathdef calls
\newcommand{\definitioncall}{Definition}
\newenvironment{stddef*}[1][]{ \begin{mathdef*}{\definitioncall}{\parent{#1}} }{ \end{mathdef*} }
\newenvironment{stddef}[1][]{ \begin{mathdef}{\definitioncall}{\parent{#1}} }{ \end{mathdef} }
% unnumerated theorem environment definition
\newenvironment{maththeorem*}[2]{
	\begin{leftbar}
	\noindent
	{ \fontfamily{ppl}\selectfont \textbf{\textsc{#1:}} } ~ #2
	\par \hfill\\ 
	\fontfamily{lmr} \fontshape{it} \selectfont
}{ 
	\end{leftbar}
}
% definitions for numerated theorem environment
\newcounter{maththeoremc}[section]
\newcommand*\maththeoremnum{\thesection.\arabic{maththeoremc}}
\renewcommand{\themaththeoremc}{\maththeoremnum}
\newenvironment{maththeorem}[2]{
	\refstepcounter{maththeoremc}
	\addcontentsline{mathlist}{figure}{\protect{\qquad\numberline{\maththeoremnum}#1 ~ #2}}
	\begin{maththeorem*}{#1 \maththeoremnum}{#2}
}{
	\end{maththeorem*}
}
% standard maththeorem calls
\newcommand{\theoremcall}{Theorem}
\newenvironment{theorem*}[1][]{ \begin{maththeorem*}{\theoremcall}{\parent{#1}} }{ \end{maththeorem*} }
\newenvironment{theorem}[1][]{ \begin{maththeorem}{\theoremcall}{\parent{#1}} }{ \end{maththeorem} }
\newcommand{\lemmacall}{Lemma}
\newenvironment{lemma*}[1][]{ \begin{maththeorem*}{\lemmacall}{\parent{#1}} }{ \end{maththeorem*} }
\newenvironment{lemma}[1][]{ \begin{maththeorem}{\lemmacall}{\parent{#1}} }{ \end{maththeorem} }
\newcommand{\propositioncall}{Proposition}
\newenvironment{proposition*}[1][]{ \begin{maththeorem*}{\propositioncall}{\parent{#1}} }{ \end{maththeorem*} }
\newenvironment{proposition}[1][]{ \begin{maththeorem}{\propositioncall}{\parent{#1}} }{ \end{maththeorem} }
\newcommand{\corollarycall}{Korollar}
\newenvironment{corollary*}[1][]{ \begin{maththeorem*}{\corollarycall}{\parent{#1}} }{ \end{maththeorem*} }
\newenvironment{corollary}[1][]{ \begin{maththeorem}{\corollarycall}{\parent{#1}} }{ \end{maththeorem} }
% q.e.d. definition
\newcommand{\qed}{ \par \hfill {\fontfamily{lmr} \fontshape{it} \selectfont \mbox{q.e.d.}} \\}
\newcommand{\qedbox}{ \par \hfill $\Box$ \\ }
% proof environment definition for theorems
\newenvironment{mathproof}[1]{
	\par \hfill\\
	\noindent
	{ \fontfamily{lmr}\selectfont \textsc{#1:}}
	\normalfont
	\small
	\begin{adjustwidth}{2em}{} 
}{ 
	\end{adjustwidth} 
	\qed
}
% standard mathproof calls
\newcommand{\proofcall}{Beweis}
\newenvironment{proof}{ \begin{mathproof}{\proofcall} }{ \end{mathproof} }
\newcommand{\proofideacall}{Beweisidee}
\newenvironment{proofidea}{ \begin{mathproof}{\proofideacall} }{ \end{mathproof} }

% new displaymath command, so that equations will not be stretched
\newcommand{\D}[1]{\mbox{$ #1 $}}
% make unnumerated equation
\newcommand{\E}[1]{\[ #1 \]}
% command for curly brackets
\newcommand{\curlb}[1]{\left\{ #1 \right\}}
% command for box brackets
\newcommand{\boxb}[1]{\left[ #1 \right]}
% command for parentheses/curved brackets
\newcommand{\curvb}[1]{\left( #1 \right)}
% command for angle brackets
\newcommand{\angleb}[1]{\left\langle #1 \right\rangle}
% command for floor brackets
\newcommand{\floorb}[1]{\left\lfloor #1 \right\rfloor}
% command for ceil brackets
\newcommand{\ceilb}[1]{\left\lceil #1 \right\rceil}
% command for creating sets
\newcommand{\set}[2]{ \left\{ #1 \enspace \middle\vert \enspace #2 \right\} }
% command for absolute value
\newcommand{\abs}[1]{\left\vert #1 \right\vert}
\newcommand{\norm}[1]{\left\| #1 \right\|}
% command for differential
\newcommand{\diff}{\mathrm{d}}
\newcommand{\Diff}{\mathrm{D}}
% command for derivative
\newcommand{\Deriv}[3][]{\Diff_{#2}^{#1}#3}
\newcommand{\deriv}[3][]{\dfrac{\diff^{#1}#2(#3)}{\diff #3^{#1}}}
% command for integral
\newcommand{\integral}[4]{\int_{#1}^{#2} #3\ \diff #4}
\newcommand{\Integral}[4]{\int\limits_{#1}^{#2} #3\ \diff #4}
% mathematical definitions (standard sets)
\newcommand{\SR}{\mathbbm{R}}
\newcommand{\SRP}{\SR^+}
\newcommand{\SRPN}{\SRP_0}
\newcommand{\SN}{\mathbbm{N}}
\newcommand{\SNN}{\SN_0}
\newcommand{\SZ}{\mathbbm{Z}}
\newcommand{\SQ}{\mathbbm{Q}}
\newcommand{\SQP}{\SQ^+}
\newcommand{\SQPN}{\SQP_0}
\newcommand{\SP}{\mathcal{P}}

% command for physical units
\newcommand{\unit}[1]{\, \text{#1}}


% package for init listings(non-formatted  text) e.g. different source codes
\usepackage{listings}


% definitions for listing colors
\definecolor{codeDarkGray}{gray}{0.2}
\definecolor{codeGray}{gray}{0.4}
\definecolor{codeLightGray}{gray}{0.9}
% predefinitions for listings
\newcommand{\listingcall}{Listing}
\newlength{\listingframemargin}
\setlength{\listingframemargin}{1em}
\newlength{\listingmargin}
\setlength{\listingmargin}{0.1\textwidth}
% \newlength{\listingwidth}
% \setlength{\listingwidth}{ ( \textwidth - \listingmargin * \real{2} + \listingframemargin * \real{2} ) }
% definitions for list of listings
\newcommand{\listoflistingscall}{\listingcall -Verzeichnis}
\newlistof{listings}{listinglist}{\listoflistingscall}
% style definition for standard code listings
\lstdefinestyle{std}{
	belowcaptionskip=0.5\baselineskip,
	breaklines=true,
	frameround=false,
	frame=tb,
	xleftmargin=0em,
	xrightmargin=0em,
	showstringspaces=false,
	showtabs=false,
	% tab=\smash{\rule[-.2\baselineskip]{.4pt}{\baselineskip}\kern.5em},
	basicstyle= \fontfamily{pcr}\selectfont\footnotesize\bfseries,
	keywordstyle= \bfseries\color{MidnightBlue}, %\color{codeDarkGray},
	commentstyle= \itshape\color{codeGray},
	identifierstyle=\color{codeDarkGray},
	stringstyle=\color{BurntOrange}, %\color{codeDarkGray},
	numberstyle=\tiny\ttfamily,
	% numbers=left,
	numbersep = 2em,
	% numberstep = 5,
	% captionpos=t,
	tabsize=4,
	backgroundcolor=\color{codeLightGray},
	framexleftmargin=\listingframemargin,
	framexrightmargin=\listingframemargin
}
% definition for unnumerated listing
\newcommand{\inputlistingn}[3][]{
	\begin{center}
		\begin{adjustwidth}{\listingmargin}{\listingmargin}
			\centerline{ {\fontfamily{lmr}\selectfont\scshape \listingcall:}\quad #2 }
			\lstinputlisting[style=std, #1]{#3}
		\end{adjustwidth}
	\end{center}
}
% definition for numerated listing
\newcounter{listingc}[section]
\newcommand*\listingnum{\thesection.\arabic{listingc}}
\renewcommand{\thelistingc}{\listingnum}
\newcommand{\inputlisting}[3][]{
	\refstepcounter{listingc}
	\addcontentsline{listinglist}{figure}{\protect{\numberline{\listingnum:} #2 } }
	\inputlistingn[#1]{#2}{#3}
}


% package for including csv-tables from file
% \usepackage{csvsimple}
% package for creating, loading and manipulating databases
\usepackage{datatool}

% package for converting eps-files to pdf-files and then include them
\usepackage{epstopdf}
% use another program (ps2pdf) for converting
% !!! important: set shell_escape=1 in /etc/texmf/texmf.cnf (Linux/Ubuntu 12.04) for allowing to use other programs
% !!!			or use the command line with -shell-escape
\epstopdfDeclareGraphicsRule{.eps}{pdf}{.pdf}{
ps2pdf -dEPSCrop #1 \OutputFile
}


% package for reference to last page (output number of last page)
\usepackage{lastpage}
% package for using header and footer
% options: automate terms of right and left marks
\usepackage[automark]{scrpage2}
% \setlength{\headheight}{4\baselineskip}
% set style for footer and header
\pagestyle{scrheadings}
% \pagestyle{headings}
% clear pagestyle for redefining
\clearscrheadfoot
% set header and footer: use <xx>head/foot[]{Text} (i...inner, o...outer, c...center, o...odd, e...even, l...left, r...right)
\ihead[]{Thermodynamik - Übung 10}
\ohead[]{Markus Pawellek \\ 144645}
\cfoot[]{\newline\newline\newline\pagemark}
% use that for mark to last page: \pageref{LastPage}
% set header separation line
\setheadsepline[\textwidth]{0.5pt}
% set foot separation line
\setfootsepline[\textwidth]{0.5pt}


\title{Thermodynamik und statistische Physik - Übung 10}
\author{Markus Pawellek}
% \date{}
	
	\newcommand*\centermathcell[1]{\omit\hfil$\displaystyle#1$\hfil\ignorespaces}

% begin the document
\begin{document}

	\section*{\centering Thermodynamik - Übung 10} % (fold)
\label{sec:thermodynamik_bung_10}

	\subsection*{Aufgabe 1} % (fold)
\label{sub:aufgabe_1}

	Sei $n\in\SN$ mit $n\geq 2$ die betrachtete Dimension des Raumes.
	Sei dann die allgemeine Koordinatentransformation von Kugelkoordinaten auf kartesische Koordinaten definiert durch
	\footnote{
		Die Idee hierfür kommt durch eine induktive Verallgemeinerung von zwei- und dreidimensionalen Kugelkoordinaten, die hier nicht erklärt werden soll.
		(Sie steht in etwas längerer Form an meinem Whiteboard.)
	}
	\[ \Phi^{n}:[0,\infty)\times[-\pi,\pi]\times\left[-\frac{\pi}{2},\frac{\pi}{2}\right]^{n-2}\longrightarrow \SR^n \]
	\begin{alignat}{3}
		&x_i:=\Phi^{n}_i (r,\alpha_1,\ldots,\alpha_{n-1}) &&:= r\sin \alpha_{i-1} &&\prod_{k=i}^{n-1} \cos \alpha_k \quad \textit{für alle } k\in\curlb{2,\ldots,n} \nonumber \\
		&x_1:=\Phi^{n}_1 (r,\alpha_1,\ldots,\alpha_{n-1}) &&:= r\cos \alpha_{1} &&\prod_{k=2}^{n-1} \cos \alpha_k \nonumber
	\end{alignat}
	Die Koordinatentransformation führt den Winkel $\vartheta$ der dreidimensionalen Kugelkoordinaten anders ein, als bekannt.
	Er wird hier um $\pi/2$ verschoben, um so einen einfacheren induktiven Vorgang zu ermöglichen.
	Die angegebenen Funktionen lassen sich außer auf einer Lebesgue-Nullmenge umkehren.
	Dabei entstehen durch Umformung die Gleichungen
	\begin{alignat}{2}
		r &= \sqrt{\sum_{i=1}^n x_i^2} \nonumber \\
		\alpha_i &= \arcsin\curvb{ \frac{x_{i+1}}{\sqrt{\sum_{k=1}^{i+1} x_k^2}} } \quad \textit{für alle } k\in\curlb{2,\ldots,n} \nonumber \\
		\alpha_1 &= 
			\begin{cases}
				\arcsin\curvb{ \frac{x_2}{\sqrt{ x_1^2 + x_2^2 }} } & :x_1 > 0 \\ 
				\arccos\curvb{ \frac{x_2}{\sqrt{ x_1^2 + x_2^2 }} } + \dfrac{\pi}{2} & :x_1 < 0,\, x_2 > 0 \\
				\arccos\curvb{ \frac{-x_2}{\sqrt{ x_1^2 + x_2^2 }} } -\dfrac{\pi}{2} & :x_1 < 0,\, x_2 < 0 
			\end{cases}
		\nonumber
	\end{alignat}
	Dies lässt sich durch vollständige Induktion nachweisen.
	Um nun die Integration über diese Koordinaten zu ermöglichen, muss die Determinante der Jacobi-Matrix von $\Phi^n$ berechnet werden.
	Dazu muss als erstes die Jacobi-Matrix dargestellt werden.
	Verwendet man ein ähnliches induktives Vorgehen, kommt man für alle $n\in\SN, n\geq2$ zu
	\[
		\Diff\Phi^{n+1}(\ldots) = 
		\left(
		\begin{array}{c|c}
			\cos\alpha_n\cdot\Diff\Phi^n(\ldots) 
			& 
			\begin{array}{c}
				-r\sin\alpha_n\cdot\Diff\Phi_{11}^n(\ldots) \\
				\vdots \\
				-r\sin\alpha_n\cdot\Diff\Phi_{n1}^n(\ldots) \\
			\end{array}\\
			\hline
			\begin{array}{rrrr}
				\sin\alpha_n & 0 & \cdots & 0 \\
			\end{array} 
			& 
			r\cos\alpha_n
		\end{array}
		\right)
	\]
	Entwickelt man nun die Determinante nach der letzten Zeile (dies ist naheliegend, da hier nur zwei Terme vorkommen, welche ungleich sind) folgt
	\begin{alignat}{2}
		\det \Diff\Phi^{n+1}(\ldots) = &(-1)^n\, r \cos\alpha_n \cdot \cos^{n}\alpha_n \, \det\Diff\Phi^n(\ldots) \nonumber \\
		& + \sin\alpha_n \cdot (-1)^{n-1} \, (-r) \, \sin\alpha_n \, \cos^{n-1}\alpha_n \, \det\Diff\Phi^n(\ldots) \nonumber
	\end{alignat}
	Für den zweiten Summanden wurde die rechte Spalte der Untermatrix zirkulär nach links vertauscht.
	So entsteht ein $(-1)^{n-1}$.
	Sowohl für die eine als auch die andere Untermatrix wurde der Satz zur Berechnung einer Determinanten, deren Spalten mit verschiedenen Skalaren multipliziert wurden, verwendet.
	So entstehen die Vorfaktoren $\cos^{n}\alpha_n$ und $(-r)\,\sin\alpha_n \, \cos^{n-1}\alpha_n$.
	Weiteres Umformen ergibt dann die Rekursionsformel der Determinanten und ein Rekursionsbeginn für $n = 2$, sodass für alle $n\in\SN, n\geq2$ gilt
	\begin{alignat}{2}
		&\det \Diff\Phi^{n+1}(\ldots) &&= (-1)^n \ r \ \cos^{n-1}\alpha_n \ \det\Diff\Phi^n(\ldots) \nonumber \\
		&\det \Diff\Phi^{2}(\ldots) &&= r \nonumber
	\end{alignat}
	Die Lösung der Rekursionsformel erfolgt durch Entwickeln und vollständige Induktion (dies wird hier nicht ausgeführt).
	Es folgt dann direkt
	\[
		\abs{ \det\Diff\Phi^{n}(\ldots) } = r^{n-1} \ \prod_{i=1}^{n-2} \cos^i\alpha_{i+1}
	\]
	Seien nun $B_R := \set{ x \in \SR^n }{ \norm{x} \leq R }$ für ein $R\in[0,\infty)$, $\lambda$ das Lebesgue-Maß und $\sigma$ das Oberflächenmaß.
	Dann gilt nach dem Satz über Koordinatentransformationen und dem Satz von Fubini/Tonelli
	\begin{alignat}{2}
		V := \lambda(B_R) = \integral{B_R}{}{}{\lambda} &= \int_0^R \int_{-\pi}^\pi \int_{-\pi/2}^{\pi/2}\cdots\int_{-\pi/2}^{\pi/2} \abs{ \det\Diff\Phi^{n}(\ldots) } \, \diff\alpha_{n-1}\cdots\diff\alpha_1\diff r \nonumber \\
		&= \dfrac{2\pi}{n}R^n \prod_{j=1}^{n-2} \integral{-\pi/2}{\pi/2}{\cos^j\alpha}{\alpha} \nonumber \\
		A := \sigma(B_R) = \integral{B_R}{}{}{\sigma} &= \int_{-\pi}^\pi \int_{-\pi/2}^{\pi/2}\cdots\int_{-\pi/2}^{\pi/2} \abs{ \det\Diff\Phi^{n}(R,\ldots) } \, \diff\alpha_{n-1}\cdots\diff\alpha_1 \nonumber \\
		&= 2\pi R^{n-1} \prod_{j=1}^{n-2} \integral{-\pi/2}{\pi/2}{\cos^j\alpha}{\alpha} \nonumber
	\end{alignat}
	Weiterhin gilt im Allgemeinen folgende Rekursionsformel für alle $n\in\SN, n>2$
	\[ \integral{-\pi/2}{\pi/2}{\cos^n\alpha}{\alpha} = \dfrac{n-1}{n} \integral{-\pi/2}{\pi/2}{\cos^{n-2}\alpha}{\alpha} \]
	Es folgt dann für alle $k\in\SN$
	\begin{alignat}{3}
		&\integral{-\pi/2}{\pi/2}{\cos^{2k}\alpha}{\alpha} &&= \dfrac{\pi}{2} &&\prod_{m=1}^{k-1} \dfrac{2m+1}{2(m+1)} \nonumber \\
		&\integral{-\pi/2}{\pi/2}{\cos^{2k-1}\alpha}{\alpha} &&= 2 &&\prod_{m=1}^{k-1} \dfrac{2m}{2m+1} \nonumber \\
		&\integral{-\pi/2}{\pi/2}{\cos\alpha}{\alpha} &&= 2 \nonumber \\
		&\integral{-\pi/2}{\pi/2}{\cos^{2}\alpha}{\alpha} &&= \dfrac{\pi}{2} \nonumber
	\end{alignat}

	\paragraph{Fall $n=2k-1$ für ein $k\in\SN$:}
	\begin{alignat}{2}
		&\prod_{j=1}^{n-2} \integral{-\pi/2}{\pi/2}{\cos^j\alpha}{\alpha} \nonumber \\
		&= \integral{-\pi/2}{\pi/2}{\cos^{2k-3}\alpha}{\alpha} \boxb{ \prod_{m=1}^{k-2} \integral{-\pi/2}{\pi/2}{\cos^{2m-1}\alpha}{\alpha} \integral{-\pi/2}{\pi/2}{\cos^{2m}\alpha}{\alpha} } \nonumber \\ 
		&= 2 \boxb{ \prod_{m=1}^{k-2} \dfrac{2m}{2m+1} } \boxb{ \prod_{m=1}^{k-2} \pi \prod_{j=1}^{m-1} \dfrac{2j}{2j+1}\dfrac{2j+1}{2(j+1)} } \nonumber \\
		&= 2^{k-1}\pi^{k-2} \boxb{ \prod_{m=1}^{k-2} \dfrac{m}{2m+1} } \cdot \dfrac{1}{(k-2)!} \nonumber \\
		&= 2^{k-1}\pi^{k-2} \prod_{m=1}^{k-2} \dfrac{1}{2m+1} \nonumber
	\end{alignat}

	\paragraph{Fall $n=2k$ für ein $k\in\SN$:}
	\begin{alignat}{2}
		\prod_{j=1}^{n-2} \integral{-\pi/2}{\pi/2}{\cos^j\alpha}{\alpha} &= \prod_{m=1}^{k-1} \integral{-\pi/2}{\pi/2}{\cos^{2m-1}\alpha}{\alpha} \integral{-\pi/2}{\pi/2}{\cos^{2m}\alpha}{\alpha} \nonumber \\
		&= \dfrac{\pi^{k-1}}{(k-1)!} \nonumber
	\end{alignat}

	Damit wurden nun die unbekannten Faktoren in den gesuchten Gleichungen bestimmt.
	Das allgemeine Volumen und der allgemeine Oberflächeninhalt folgt nun durch Einsetzen.
	\begin{alignat}{2}
		V &=
		\begin{cases}
			\frac{\pi^k}{k!}R^n & :n=2k, k\in\SN \\
			2^{k}\pi^{k-1}R^n \prod_{m=1}^{k-1} \frac{1}{2m+1} & :n=2k-1, k\in\SN \\
		\end{cases}
		\nonumber \\
		A &=
		\begin{cases}
			\frac{2\pi^k}{(k-1)!}R^{n-1} & :n=2k, k\in\SN \\
			2^{k}\pi^{k-1}R^{n-1} \prod_{m=1}^{k-2} \frac{1}{2m+1} & :n=2k-1, k\in\SN \\
		\end{cases}
		\nonumber
	\end{alignat}
	\newpage
	Sei nun $R>\varepsilon > 0$.
	Dann besitzt die äußere Kugelschale mit der Schichtdicke $\varepsilon$ das Volumen $\Delta V$, wenn $\xi$ die Funktion der Vorfaktoren des Volumens darstellt.
	\begin{alignat}{2}
		&\Delta V &&= \xi(n) \boxb{ R^n - (R-\varepsilon)^n } \nonumber \\
		\Rightarrow &\frac{\Delta V}{V} &&= 1 - \underbrace{\curvb{ 1-\frac{\varepsilon}{R} }^n }_{\mathclap{\longrightarrow 0, n\longrightarrow \infty}} \longrightarrow 1, \ n\longrightarrow\infty \nonumber
	\end{alignat}
	Damit befindet sich für große Dimensionen $n$ und für alle $\varepsilon>0$ fast das gesamte Volumen in der äußeren Schale.

% subsection aufgabe_1 (end)
	\newpage
	\subsection*{Aufgabe 2} % (fold)
\label{sub:aufgabe_2}

	Sei $n\in\SN$ mit $n>1$.
	Sei weiterhin \D{Z=(1,2,\ldots,n)} eine Zerlegung von $[1,n]$.
	Die Funktion $\ln n!$ ist monoton steigend und positiv für alle $n\in\SN$ mit $n\geq 1$.
	Es gilt nach Gesetzen des Logarithmus
	\[ \ln n! = \sum_{i=1}^{n}\ln{i} \]
	Daraus folgt dann durch Betrachtung der Ober- und Untersummen
	\[ \integral{1}{n}{\ln x}{x} \ \leq \ \sum_{i=1}^n \ln i \ \leq \ \integral{1}{n+1}{\ln x}{x} \]
	Rückumformung und Anwendung des Hauptsatzes der Integral- und Differentialrechnung resultiert dann in
	\begin{alignat}{6}
		&&n\ln n - n+1 \ \leq && \centermathcell{\ln n!} &\leq \ (n+1)\ln(n+1) -n \nonumber \\
		\Rightarrow &&\ln n -\frac{n-1}{n} \ \leq &&\centermathcell{\frac{\ln n!}{n}} &\leq \ \frac{n+1}{n}\ln(n+1) - 1 \nonumber \\
		\Rightarrow &&\underbrace{-\frac{n-1}{n}}_{\mathclap{\longrightarrow -1,\ n \longrightarrow \infty}} \ \leq &&\centermathcell{\ \frac{\ln n!}{n} - \ln n} \ &\leq \ \underbrace{\frac{n+1}{n}\ln(n+1) -\ln n - 1}_{\mathclap{= \ln\frac{n+1}{n} + \frac{\ln(n+1)}{n} - 1 \longrightarrow -1,\ n \longrightarrow \infty}} \nonumber
	\end{alignat}
	Nach dem Sandwich-Theorem für Folgen folgt also
	\[ \frac{\ln n!}{n} - \ln n  \longrightarrow -1,\ n \longrightarrow \infty \]
	\qed

% subsection aufgabe_2 (end)
	\newpage
	\subsection*{Aufgabe 3} % (fold)
\label{sub:aufgabe_3}

	Seien $n\in\SN$ die Anzahl der Tage, an denen Personen Geburtstag haben können, und $k\in\SN$ mit $k<n$ die Anzahl der betrachteten Personen.
	Nimmt man die Gleichverteilung für Geburtstage an, so folgt für das Wahrscheinlichkeitsmaß
	\footnote{Der Wahrscheinlichkeitsraum soll hier nicht genauer angegeben werden.}
	\[ P:\SP\curvb{ \curlb{1,\ldots,n}^k }\longrightarrow [0,1] ,\qquad P((n_1,\ldots,n_k)) = \frac{1}{n^k} \]
	Das komplementäre Ereignis von 
	\[ \textit{Mindestens zwei Personen haben am gleichen Tag Geburtstag.} \]
	ist
	\[ \textit{Alle Personen haben an verschiedenen Tagen Geburtstag.} \]
	Dieses komplementäre Ereignis kann gerade durch folgende Menge beschrieben werden.
	$$ A:=\set{ (n_1,\ldots,n_k)\in\curlb{1,\ldots,n}^k }{ n_i \neq n_j \textit{ für alle } i,j\in\curlb{1,\ldots,k},i\neq j } $$
	Um nun die Anzahl der Elemente von $A$ zu bestimmen, bestimmt man induktiv die Anzahl der Möglichkeiten der einzelnen $n_i$.
	Beginnt man der Benennung entsprechend mit $n_1$, so können dessen Werte beliebig gewählt werden, da die anderen Werte noch nicht festgelegt wurden.
	Damit gibt es für $n_1$ also $n$ Möglichkeiten.
	Folglich kann aber $n_2$ nicht mehr beliebig gewählt werden, weil bereits $n_1$ einen der beliebigen $n$ Werte angenommen hat.
	Für $n_2$ gibt es also genau eine Lösung weniger und damit genau $n-1$ Möglichkeiten.
	Wählt man nun ein ganzzahliges $i<k$, sodass die Anzahl der Möglichkeiten für $n_i$ bekannt ist und mit $N_i$ bezeichnet wird, dann müssen sich die Möglichkeiten für $n_{i+1}$ durch $N_i -1$ ergeben, da durch $n_i$ einer der noch möglichen $N_i$ Werte besetzt wird.
	Die Anzahl der Gesamtmöglichkeiten ergibt sich dann durch Multiplikation der induktiv bestimmten Werte.
	$$ \#A = n\cdot(n-1)\cdot(n-2)\cdot\cdots\cdot(n-k+2)\cdot(n-k+1) = \frac{n!}{(n-k)!} $$
	Es folgt dann
	$$ P(A) = \frac{\#A}{n^k} = \frac{n!}{n^k(n-k)!} $$
	Nach Gesetzen der Wahrscheinlichkeitsrechnung  ergibt sich dann die Wahrscheinlichkeit des eigentlichen Ereignisses durch
	$$ \boxed{P(A^c) = 1- \frac{n!}{n^k(n-k)!}} $$
	\begin{description}
		\item[$n=365,k=20$:]
			$$ P(A^c) \approx 0.411 $$
		\item[$n=365,k=60$:]
			$$ P(A^c) \approx 0.994 $$ 
	\end{description}

% subsection aufgabe_3 (end)

% section thermodynamik_bung_10 (end)

\end{document}